% PDF build: pdflatex -> bibtex -> pdflatex
\documentclass[11pt,            % 8, 9, 10, 11 (default), 12, 14, 17, and 20pt
               aspectratio=169, % 43 for 4:3, 169 for 16:9, 1610 for 16:10
               xcolor=svgnames,
               t                % top alignment for vertical alignment
               ]{beamer}

\usepackage{graphicx}
\usepackage{colortbl}
\usepackage[square]{natbib}             % bibliography style package
\usepackage{incgraph}
\usepackage{tikz}
\usepackage{listings}
\usepackage{multimedia}
\usepackage[bahasa]{babel}
\usepackage{amsthm,pifont}
\usepackage{lipsum}

% EDIT file preamble.tex
% EDIT THIS FILE (MIHpreamble.tex)
\lstdefinestyle{CStyle}{
    backgroundcolor=\color{white},
    commentstyle=\itshape\color{DarkGreen},
    keywordstyle=\bfseries\color{red},
    numberstyle=\tiny\color{Gray},
    stringstyle=\color{DodgerBlue},
    basicstyle=\ttfamily,
    breakatwhitespace=false,
    breaklines=true,
    captionpos=b,
    keepspaces=true,
    numbers=left,
    numbersep=5pt,
    showspaces=false,
    showstringspaces=false,
    showtabs=false,
    tabsize=2,
    language=C
}
\lstdefinestyle{PyStyle}{
    backgroundcolor=\color{white},
    commentstyle=\itshape\color{DarkGreen},
    keywordstyle=\bfseries\color{red},
    numberstyle=\tiny\color{Gray},
    stringstyle=\color{DodgerBlue},
    basicstyle=\ttfamily,
    breakatwhitespace=false,
    breaklines=true,
    captionpos=b,
    keepspaces=true,
    numbers=left,
    numbersep=5pt,
    showspaces=false,
    showstringspaces=false,
    showtabs=false,
    tabsize=2,
    language=Python
}
\lstdefinestyle{FortranStyle}{
    backgroundcolor=\color{white},
    commentstyle=\itshape\color{DarkGreen},
    keywordstyle=\bfseries\color{red},
    numberstyle=\tiny\color{Gray},
    stringstyle=\color{DodgerBlue},
    basicstyle=\ttfamily,
    breakatwhitespace=false,
    breaklines=true,
    captionpos=b,
    keepspaces=true,
    numbers=left,
    numbersep=5pt,
    showspaces=false,
    showstringspaces=false,
    showtabs=false,
    tabsize=2,
    language=Fortran
}
\usetikzlibrary{shadows,calc}
\bibpunct{(}{)}{;}{a}{}{,}              % bibliography style format:
%         author [year]
% If you want to use Bahasa Indonesia
\usepackage[bahasa]{babel}
\uselanguage{Bahasa}
\languagepath{Bahasa}
\deftranslation[to=Bahasa]{Theorem}{Teorema}
\deftranslation[to=Bahasa]{Corollary}{Akibat}
\deftranslation[to=Bahasa]{Definition}{Definisi}
\deftranslation[to=Bahasa]{Example}{Contoh}
\deftranslation[to=Bahasa]{Proof}{Bukti}

% ITB COLORS:
\definecolor{ITB}{RGB}{40, 56,144} % note the uppercase RGB
\definecolor{G10}{RGB}{16,116,188} % note the uppercase RGB

\usetheme{Warsaw}
\useoutertheme{infolines,shadow}
\useinnertheme{rounded}
\usefonttheme{professionalfonts,structurebold}
\usecolortheme[named=ITB]{structure}
\setbeamertemplate{items}[ball]                  %default,ball,circle,rectangle
\setbeamertemplate{blocks}[rounded][shadow=true]
\setbeamertemplate{navigation symbols}{}
\setbeamertemplate{theorems}[numbered]
\setbeamertemplate{caption}[numbered]
%\setbeamertemplate{frametitle}[default][left]   %unfortunately, shadow disappear
\setbeamercolor{normal text}{fg=black}
%\setbeamertemplate{background canvas}{\includegraphics
%    [width=\paperwidth,height=\paperheight]{whiteLandscape.jpg}}
\setbeamercovered{transparent}
\usebackgroundtemplate{%
    \begin{tikzpicture}[remember picture,overlay]\node[opacity=0.01,inner sep=0] at (current page.center) {\includegraphics[scale=0.15]{logoITB}};\end{tikzpicture} % best opacity = 0.025
}
\hypersetup{colorlinks=true,allcolors=FireBrick,bookmarksopenlevel=1}
% \newtheorem{teorema}{Teorema} % theorem environment using Bahasa
% SOME NEW THEOREM/BLOCK ENVIRONMENTS %%%%%%%%%%%%%%%%%%%%%%%%%%%%%%%%%%%%%%%%%%%%
\newtheorem{instance}{Example}
\newcommand*{\theorembreak}{\usebeamertemplate{theorem end}\framebreak\usebeamertemplate{theorem begin}}
% Example or instance %%%%%%%%%%%%%%%%%%%%%%%%%%%%%%%%%%%%%%%%%%%%%%%%%%%%%%%%%%%%
\BeforeBeginEnvironment{instance}{
    \setbeamercolor{block title}{use=example text,fg=white,bg=Brown!50!black}
    \setbeamercolor{block body}{parent=normal text,use=block title example,bg=block title example.bg!10!bg}
}
\AfterEndEnvironment{instance}{
    \setbeamercolor{block title}{use=structure,fg=white,bg=structure.fg!75!black}
    \setbeamercolor{block body}{parent=normal text,use=block title,bg=block title.bg!10!bg}
}
\newtheorem{exercise}{Exercise}
% Exercise %%%%%%%%%%%%%%%%%%%%%%%%%%%%%%%%%%%%%%%%%%%%%%%%%%%%%%%%%%%%%%%%%%%%%%%
\BeforeBeginEnvironment{exercise}{
    \setbeamercolor{block title}{use=example text,fg=white,bg=DarkKhaki!50!black}
    \setbeamercolor{block body}{parent=normal text,use=block title example,bg=block title example.bg!10!bg}
}
\AfterEndEnvironment{exercise}{
    \setbeamercolor{block title}{use=structure,fg=white,bg=structure.fg!75!black}
    \setbeamercolor{block body}{parent=normal text,use=block title,bg=block title.bg!10!bg}
}
\newtheorem{assignment}{Assignment/Homework}
% Exercise %%%%%%%%%%%%%%%%%%%%%%%%%%%%%%%%%%%%%%%%%%%%%%%%%%%%%%%%%%%%%%%%%%%%%%%
\BeforeBeginEnvironment{assignment}{
    \setbeamercolor{block title}{use=example text,fg=white,bg=black}
    \setbeamercolor{block body}{parent=normal text,use=block title example,bg=block title example.bg!10!bg}
}
\AfterEndEnvironment{assignment}{
    \setbeamercolor{block title}{use=structure,fg=white,bg=structure.fg!75!black}
    \setbeamercolor{block body}{parent=normal text,use=block title,bg=block title.bg!10!bg}
}
\newtheorem{algoblock}{Algorithm}
% Example or instance %%%%%%%%%%%%%%%%%%%%%%%%%%%%%%%%%%%%%%%%%%%%%%%%%%%%%%%%%%%%
\BeforeBeginEnvironment{algoblock}{
    \setbeamercolor{block title}{use=example text,fg=white,bg=G10}
    \setbeamercolor{block body}{parent=normal text,use=block title example,bg=block title example.bg!10!bg}
}
\AfterEndEnvironment{algoblock}{
    \setbeamercolor{block title}{use=structure,fg=white,bg=structure.fg!75!black}
    \setbeamercolor{block body}{parent=normal text,use=block title,bg=block title.bg!10!bg}
}
%%%%%%%%%%%%%%%%%%%%%%%%%%%%%%%%%%%%%%%%%%%%%%%%%%%%%%%%%%%%%%%%%%%%%%%%%%%%%%%%%%

% Drop image shadow with gaussian blur (using Tikz, not fancybox) %%%%%%%%%%%%%%%%
% some parameters for customization
\def\shadowshift{3pt,-3pt}
\def\shadowradius{6pt}
\colorlet{innercolor}{black!60}
\colorlet{outercolor}{gray!05}
% this draws a shadow under a rectangle node
\newcommand\drawshadow[1]{
    \begin{pgfonlayer}{shadow}
        \shade[outercolor,inner color=innercolor,outer color=outercolor] ($(#1.south west)+(\shadowshift)+(\shadowradius/2,\shadowradius/2)$) circle (\shadowradius);
        \shade[outercolor,inner color=innercolor,outer color=outercolor] ($(#1.north west)+(\shadowshift)+(\shadowradius/2,-\shadowradius/2)$) circle (\shadowradius);
        \shade[outercolor,inner color=innercolor,outer color=outercolor] ($(#1.south east)+(\shadowshift)+(-\shadowradius/2,\shadowradius/2)$) circle (\shadowradius);
        \shade[outercolor,inner color=innercolor,outer color=outercolor] ($(#1.north east)+(\shadowshift)+(-\shadowradius/2,-\shadowradius/2)$) circle (\shadowradius);
        \shade[top color=innercolor,bottom color=outercolor] ($(#1.south west)+(\shadowshift)+(\shadowradius/2,-\shadowradius/2)$) rectangle ($(#1.south east)+(\shadowshift)+(-\shadowradius/2,\shadowradius/2)$);
        \shade[left color=innercolor,right color=outercolor] ($(#1.south east)+(\shadowshift)+(-\shadowradius/2,\shadowradius/2)$) rectangle ($(#1.north east)+(\shadowshift)+(\shadowradius/2,-\shadowradius/2)$);
        \shade[bottom color=innercolor,top color=outercolor] ($(#1.north west)+(\shadowshift)+(\shadowradius/2,-\shadowradius/2)$) rectangle ($(#1.north east)+(\shadowshift)+(-\shadowradius/2,\shadowradius/2)$);
        \shade[outercolor,right color=innercolor,left color=outercolor] ($(#1.south west)+(\shadowshift)+(-\shadowradius/2,\shadowradius/2)$) rectangle ($(#1.north west)+(\shadowshift)+(\shadowradius/2,-\shadowradius/2)$);
        \filldraw ($(#1.south west)+(\shadowshift)+(\shadowradius/2,\shadowradius/2)$) rectangle ($(#1.north east)+(\shadowshift)-(\shadowradius/2,\shadowradius/2)$);
    \end{pgfonlayer}
}
% create a shadow layer, so that we don't need to worry about overdrawing other things
\pgfdeclarelayer{shadow}
\pgfsetlayers{shadow,main}
\newsavebox\mybox
\newlength\mylen
\newcommand\shadowimage[3][]{%
    \setbox0=\hbox{\includegraphics[#1]{#2}}
    \setlength\mylen{\wd0}
    \ifnum\mylen<\ht0
    \setlength\mylen{\ht0}
    \fi
    \divide \mylen by 120
    \def\shadowshift{\mylen,-\mylen}
    \def\shadowradius{\the\dimexpr\mylen+\mylen+\mylen\relax}
    \begin{figure}
        \begin{tikzpicture}
            \node[anchor=south west,inner sep=0] (image) at (0,0) {\includegraphics[#1]{#2}};
            \drawshadow{image}
        \end{tikzpicture}
        \caption{#3}
\end{figure}}
%%%%%%%%%%%%%%%%%%%%%%%%%%%%%%%%%%%%%%%%%%%%%%%%%%%%%%%%%%%%%%%%%%%%%%%%%%%%%%%%%%

% MAIN BODY  %%%%%%%%%%%%%%%%%%%%%%%%%%%%%%%%%%%%%%%%%%%%%%%%%%%%%%%%%%%%%%%%%%%%%
\begin{document}

% (1a) MIH CUSTOM TITLE STYLE (RECOMMENDED) %%%%%%%%%%%%%%%%%%%%%%%%%%%%%%%%%%%%%%
\title{\LaTeX\ untuk Tugas Akhir}
\subtitle{Sebuah Pengantar}
\author[M I Hakim]{Dr. Muhamad Irfan Hakim}
\titlegraphic{\includegraphics[width=2cm]{logoITB.png}} % logo as watermark is also provided
\logo{\includegraphics[scale=0.02]{logoITB}} % logo as watermark is also provided
\institute[FMIPA -- ITB]{Prodi Astronomi\\
    FMIPA -- ITB\\\bigskip
    \textbf{AS4091 Tugas Akhir I \& AS4092 Tugas Akhir II}}
\date{\number\year}

\setbeamercolor{ITBtitle}{fg=ITB,bg=White}
\makeatletter
\setbeamertemplate{title page}{
    \vbox{}
    \vfill
    \begingroup
    %\centering
    \hfill
    {\usebeamercolor[fg]{titlegraphic}\inserttitlegraphic\par}
    \begin{beamercolorbox}[sep=8pt,right]{ITBtitle} % ITBtitle <--> title
        \usebeamerfont{title}\usebeamercolor[bg]{title}\inserttitle\par%
        \ifx\insertsubtitle\@empty%
        \else%
        \vskip0.25em%
        {\usebeamerfont{subtitle}\usebeamercolor[bg]{subtitle}\insertsubtitle\par}%
        \fi%
    \end{beamercolorbox}%
    \vskip1em\par
    \begin{beamercolorbox}[sep=8pt,right]{author}
        \usebeamerfont{author}\itshape\insertauthor
    \end{beamercolorbox}
    \begin{beamercolorbox}[sep=8pt,right]{institute}
        \usebeamerfont{institute}\insertinstitute
    \end{beamercolorbox}
    \vspace{-1 true cm}
    \includegraphics[scale=0.3]{MIHakim-cropped}\hfill\usebeamerfont{date}\insertdate\hspace{8pt}
    \endgroup
    \vfill
}
\makeatother
\begin{frame}[plain,label=home]
\titlepage
\pdfbookmark[0]{Beamer}{home}
\end{frame}
%%%%%%%%%%%%%%%%%%%%%%%%%%%%%%%%%%%%%%%%%%%%%%%%%%%%%%%%%%%%%%%%%%%%%%%%%%%%%%%%%%

% (1b) MIH CUSTOM TITLE STYLE %%%%%%%%%%%%%%%%%%%%%%%%%%%%%%%%%%%%%%%%%%%%%%%%%%%%
%\begin{frame}[plain,label=home]
%\begin{flushright}
%{\Huge\bfseries Beamer}
%
%\vspace{0.25 true cm}
%
%{\LARGE\bfseries Template}
%
%\vspace{0.75 true cm}
%
%{\Large\itshape Dr. Muhamad Irfan Hakim}
%
%\vspace{0.5 true cm}
%
%Astronomy Research Division\\
%FMIPA -- ITB
%
%\bigskip
%
%\textbf{ASnnnn Coursename}
%\end{flushright}
%
%\includegraphics[scale=0.3]{MIHakim-cropped}\hfill{\bfseries\number\year}
%
%\vspace{0.05 true cm}
%
%\hrule
%\pdfbookmark[0]{Beamer}{home}
%\end{frame}
%%%%%%%%%%%%%%%%%%%%%%%%%%%%%%%%%%%%%%%%%%%%%%%%%%%%%%%%%%%%%%%%%%%%%%%%%%%%%%%%%%

% (1c) DEFAULT TITLE STYLE %%%%%%%%%%%%%%%%%%%%%%%%%%%%%%%%%%%%%%%%%%%%%%%%%%%%%%%
%\title{Beamer}
%\subtitle{Template}
%\author[M I Hakim]{\itshape Dr. Muhamad Irfan Hakim}
%\titlegraphic{
%    \includegraphics[width=2cm]{logoITB.png}
%} % logo as watermark is also provided
%\logo{\includegraphics[scale=0.02]{logoITB}} % logo as watermark is also provided
%\institute[FMIPA -- ITB]{Astronomy Research Division\\
%                         FMIPA -- ITB\\\bigskip
%                         \textbf{ASnnnn Coursename}}
%\date{\number\year}
%\begin{frame}[plain,label=home]
%\maketitle
%\pdfbookmark[0]{Beamer}{home}
%\end{frame}
%%%%%%%%%%%%%%%%%%%%%%%%%%%%%%%%%%%%%%%%%%%%%%%%%%%%%%%%%%%%%%%%%%%%%%%%%%%%%%%%%%

%%%%%%%%%%%%%%%%%%%%%%%%%%%%%%%%%%%%%%%%%%%%%%%%%%%%%%%%%%%%%%%%%%%%%%%%%%%%%%%%%%
%=======================================================================
\begin{frame}{Disclaimer}
\label{disclaimer}
Struktur materi di slide ini belum dianggap lengkap, dan masih akan diatur ulang dan atau diupdate ke \href{https://github.com/irfan200867/LaTeX-dan-TA-Tesis}{github.com/irfan200867/LaTeX-dan-TA-Tesis}
\pdfbookmark[1]{Disclaimer}{disclaimer}
\end{frame}
%=======================================================================

%=======================================================================
\begin{frame}{Notes}
\label{notes}
--
\pdfbookmark[1]{Notes}{notes}
\end{frame}
%=======================================================================
%%%%%%%%%%%%%%%%%%%%%%%%%%%%%%%%%%%%%%%%%%%%%%%%%%%%%%%%%%%%%%%%%%%%%%%%%%%%%%%%%%

% (2) TABLE OF CONTENTS %%%%%%%%%%%%%%%%%%%%%%%%%%%%%%%%%%%%%%%%%%%%%%%%%%%%%%%%%%
\begin{frame}{\contentsname}
\label{contents}
\tableofcontents
\pdfbookmark[1]{\contentsname}{contents}
\end{frame}
%%%%%%%%%%%%%%%%%%%%%%%%%%%%%%%%%%%%%%%%%%%%%%%%%%%%%%%%%%%%%%%%%%%%%%%%%%%%%%%%%%

\section{Sumber Primer}

\begin{frame}{Sumber Primer \TeX\ , \LaTeX\ dan sekawannya}
\begin{block}{Referensi}
\begin{itemize}
\item \TeX\ \textbf{\underline{IS} the source}. See \href{https://tug.org/}{tug.org}
\item \LaTeX\ adalah salah satu \textbf{varian \underline{turunan}} dari \TeX
\item TeXLive, MikTeX, \ldots adalah contoh nama-nama untuk \textbf{\underline{distribusi}} \TeX~dan~\LaTeX
\item TeXstudio, TeXniccenter, \ldots adalah contoh nama-nama \textbf{\underline{editor}} \LaTeX
\item Saran saya? Mulailah install \textbf{TeXLive}$+$\textbf{TeXstudio}
\item Bisakah menulis dalam format LaTeX tanpa install TeXLive dsb.? \textbf{Bisa}. Gunakan saja fasilitas di laman  \href{https://www.overleaf.com}{overleaf.com}
\item Bisakah konversi dari format lain? \textbf{Bisa}. Setelah konversi, file \LaTeX~disunting seperlunya.
\begin{itemize}
    \item Jupyter Notebook: \textit{via} \textbf{jupyter nbconvert} di \underline{\itshape python environment} seperti Anaconda Prompt
    \item MS Word (DOCX): \textit{via} \textbf{pandoc}
\end{itemize}
\item \textit{Need help}, mau \textit{ngoprek}, \textit{custom}: \href{https://tug.org/}{tug.org}, \href{https://www.overleaf.com}{overleaf.com}, TeXstudio LaTeX Reference/User Manual, \href{https://tex.stackexchange.com/}{tex.stackexchange.com}, etc.
\end{itemize}
\end{block}
\end{frame}

\begin{frame}[plain]
\incgraph[documentpaper][width=\paperwidth,height=0.97\paperheight]{tugorg.png}
\end{frame}

\begin{frame}[plain]
\incgraph[documentpaper][width=\paperwidth,height=0.97\paperheight]{overleaf.png}
\end{frame}

\begin{frame}[plain]
\incgraph[documentpaper][width=\paperwidth,height=0.97\paperheight]{texstackexchange.png}
\end{frame}

\section{Struktur pada Kelas Dokumen}

\begin{frame}{Struktur pada Kelas Dokumen}
\begin{block}{Kelas dan Struktur}
\begin{itemize}
    \item Contoh kelas-kelas: \texttt{article}, \texttt{report}, \texttt{book}, \texttt{beamer} (untuk presentasi)
    \item Struktur dan hirarki:

\bigskip

        \label{hirarki}

    \begin{center}
    \begin{tabular}{|c|l|c|c|c|c|}
    \hline
    \textbf{Urutan} & \textbf{Bagian dokumen} & \texttt{book} & \texttt{report} & \texttt{article} & \texttt{beamer}\\\hline\hline
    1 & \texttt{part} & \ding{51} & \ding{51} & \ding{51} & \\
    2 & \texttt{chapter} & \ding{51} & \ding{51} & & \\
    3 & \texttt{section} & \ding{51} & \ding{51} & \ding{51} & \ding{51}\\
    4 & \texttt{subsection} & \ding{51} & \ding{51} & \ding{51} & \ding{51}\\
    5 & \texttt{subsubsection} & \ding{51} & \ding{51} & \ding{51} & \ding{51}\\\hline
    \end{tabular}
    \end{center}

\end{itemize}

\bigskip

\textbf{Bisa diperiksa di TeXstudio LaTeX Reference/User Manual.}

\end{block}
\end{frame}

\section{Strategi Draft Laporan TA/Tesis \& Presentasi}

\begin{frame}{Strategi Draft Laporan TA/Tesis \& Presentasi dalam \texttt{beamer}}
\begin{block}{Saran}
\begin{enumerate}
\item Teruskan saja dulu \textit{menulis dengan piranti paling dikuasai}\label{nulisAja}
\item \textit{Pahami struktur dasar} dokumen TA/Tesis versi \LaTeX
\item Bila langkah~(\ref{nulisAja}) tidak dalam \LaTeX, \textit{konversikan} ke \LaTeX\ \textit{saat
      sudah dianggap selesai}
\item \textit{Sunting sesuai dengan standard} dokumen \LaTeX\ dan TA/Tesis, termasuk bibliografi.
\item \textbf{Konversi ke PDF} \textit{via} \texttt{pdflatex} dan \texttt{bibtex}, yaitu:
      \texttt{pdflatex} $ \rightarrow $ \texttt{bibtex} $ \rightarrow $ \texttt{pdflatex}
\item Untuk slide presentasi, \textbf{ekstrak isi dokumen} \LaTeX\ \textbf{hanya pada bagian-bagian penting} dan menjadi \textit{etalase} TA/Tesis, serta \textit{\bfseries downgrade} \textbf{hirarkinya mengikuti standard }\texttt{beamer} (lihat halaman~\pageref{hirarki})
\end{enumerate}
\end{block}
\end{frame}

\begin{frame}[plain]
    \vfill
    \centering{\Huge\bfseries Now, \LaTeX\ \ldots}
    \vfill
\end{frame}

\section{Enam Tanda Baca Terpenting}

\begin{frame}
\begin{block}{Enam Tanda Baca Terpenting}
\bigskip

\centering{\hspace{0.2cm}$\backslash$\hfill\%\hfill\{\hfill\}\hfill\ [\hfill\ ]\hspace{0.2cm}}

\bigskip
\end{block}

\begin{block}{Peruntukkannya}
``$ \backslash $'' untuk \textit{conserved words} (misal perintah, parameter) di \TeX\ atau \LaTeX. Sebagian perintah \TeX/\LaTeX\ hanya untuk \textbf{\underline{MODA TEKS}} atau hanya untuk \textbf{\underline{MODA MATEMATIKA}}.

\bigskip

``\%'' di suatu baris berguna untuk mencegah agar apapun \textbf{di sebelah kanan tanda \% tidak dieksekusi} \TeX\ atau \LaTeX.

\bigskip

``\{'' dan ``\}'' untuk penanda blok \textbf{argumen wajib}

\bigskip

``['' dan ``]'' untuk penanda blok \textbf{argumen opsional}
\end{block}

\end{frame}

\section{Penanda Block/Environment}

\begin{frame}[containsverbatim]{Penanda Block/Environment}
\begin{block}{Penanda Block/Environment pada \LaTeX}
\begin{verbatim}
$  ... $           % untuk satu ekspresi matematika DI DALAM PARAGRAF
\[ ... \]          % untuk satu ekspresi matematika/rumus tanpa nomor
   \begin{equation}
    ...            % untuk satu rumus bernomor
   \end{equation}

\begin{equation*}
    ...            % untuk satu rumus tak bernomor
\end{equation*}

\begin{eqnarray}
    ...            % untuk rangkaian baris rumus
\end{eqnarray}
\end{verbatim}
\end{block}

\end{frame}

\begin{frame}[containsverbatim]{Penanda Block/Environment}

    \begin{block}{Penanda Block/Environment pada \LaTeX}
        \begin{verbatim}
            \begin{figure}
                ...            % untuk sisip gambar
            \end{figure}

            \begin{table}
                ...            % untuk menyusun tabel + caption
            \end{table}

            \begin{tabular}{|c|c|}
                ...            % contoh desain tabel 2 kolom rapat tengah
            \end{tabular}
        \end{verbatim}
    \end{block}
\ldots\textbf{dan seterusnya} (editor \LaTeX\ yang baik dapat memandu pilihan \textit{environment}) \ldots
\end{frame}

%%%%%%%%%%%%%%%%%%%%%%%%%%%%%%%%%%%%%%%%%%%%%%%%%%%%%%%%%%%%%%%%%%%%%%%%%%%%%%%%%%
\section{Matematika}
%=======================================================================
\begin{frame}[containsverbatim,allowframebreaks]{Matematika}
\label{math}

Ekspresi matematika $\int x dx$  (ditulis \verb|$\int x dx$|) bergabung dengan naskah dalam satu paragraf.

\bigskip

Sebaris persamaan tak bernomor (terpisah dari paragraf):

\[ \alpha + \beta = 1 \]

\begin{block}{}
\begin{verbatim}
                      \[ \alpha + \beta = 1 \]
\end{verbatim}
\end{block}

\bigskip

Dapat dikatakan bahwa menuliskan $\int x dx$ dan $\alpha + \beta = 1$ \textbf{hampir setara dengan membacanya}, kecuali bahwa ada tambahan tanda-tanda baca sebagai penambahnya. Secara umum, demikian pula sebagian (cukup) besar perintah di \LaTeX~bersifat serupa.

\newpage
Sebaris persamaan bernomor (terpisah dari paragraf):

\begin{equation}
\label{sqrtpi}
\int_{-\infty}^{\infty} e^{-x^2} \, dx = \sqrt{\pi}
\end{equation}

\begin{block}{}
\begin{verbatim}
\begin{equation}
\label{sqrtpi}
\int_{-\infty}^{\infty} e^{-x^2} \, dx = \sqrt{\pi}
\end{equation}
\end{verbatim}
\end{block}

\bigskip

Penambahan ``\verb|\label{sqrtpi}|'' berguna untuk \textbf{rujukan silang} (\textit{cross reference}).


\newpage
Barisan persamaan (bernomor dan tak bernomor, terpisah dari paragraf):

\begin{eqnarray}
\sin^2\,x + \cos^2\,x &=& 1\\
(a + b)^2             &=& a^2 + 2ab + b^2\nonumber\\
\alpha &=& \beta + \delta
\end{eqnarray}

\begin{block}{}
\begin{verbatim}
\begin{eqnarray}
\sin^2\,x + \cos^2\,x &=& 1\\
(a + b)^2             &=& a^2 + 2ab + b^2\nonumber\\
\alpha &=& \beta + \delta
\end{eqnarray}
\end{verbatim}
\end{block}

\newpage

Barisan persamaan tak bernomor dan terpisah dari paragraf:

\begin{eqnarray*}
\sin^2\,x + \cos^2\,x &=& 1\\
(a + b)^2             &=& a^2 + 2ab + b^2\\
\alpha &=& \beta + \delta
\end{eqnarray*}

\begin{block}{}
\begin{verbatim}
\begin{eqnarray*}
\sin^2\,x + \cos^2\,x &=& 1\\
(a + b)^2             &=& a^2 + 2ab + b^2\\
\alpha &=& \beta + \delta
\end{eqnarray*}
\end{verbatim}
\end{block}

\end{frame}
%=======================================================================
%%%%%%%%%%%%%%%%%%%%%%%%%%%%%%%%%%%%%%%%%%%%%%%%%%%%%%%%%%%%%%%%%%%%%%%%%%%%%%%%%%

%%%%%%%%%%%%%%%%%%%%%%%%%%%%%%%%%%%%%%%%%%%%%%%%%%%%%%%%%%%%%%%%%%%%%%%%%%%%%%%%%%
\section{Daftar (List)}
%=======================================================================
\begin{frame}[containsverbatim,allowframebreaks]{Daftar (List)}
\label{list}

Daftar tak bernomor urut (\textit{itemized list}):

\begin{itemize}
\item itemized item 1
\item itemized item 2
\item itemized item 3
\end{itemize}

\begin{block}{}
\begin{verbatim}
\begin{itemize}
\item itemized item 1
\item itemized item 2
\item itemized item 3
\end{itemize}
\end{verbatim}
\end{block}
\newpage
Daftar bernomor urut (\textit{enumerated list}):

\begin{enumerate}
\item Three
\item One
\item Two
\end{enumerate}

\begin{block}{}
\begin{verbatim}
\begin{enumerate}
\item Three
\item One
\item Two
\end{enumerate}
\end{verbatim}
\end{block}

\end{frame}
%=======================================================================
%%%%%%%%%%%%%%%%%%%%%%%%%%%%%%%%%%%%%%%%%%%%%%%%%%%%%%%%%%%%%%%%%%%%%%%%%%%%%%%%%%

%%%%%%%%%%%%%%%%%%%%%%%%%%%%%%%%%%%%%%%%%%%%%%%%%%%%%%%%%%%%%%%%%%%%%%%%%%%%%%%%%%
%\section{Transisi Salindia}
%=======================================================================
%\begin{frame}{Transisi Salindia}
%% transblindshorizontal/vertical, transboxin/out, transdissolve, transglitter
%% transslipverticalin/out, transhorizontalin/out, transwipe, transduration{2}
%\pause
%\transboxin<1>
%Frame Body Text (transboxin). \pause\transboxout<2>Frame Body Text (transboxout).
%
%\pause\transblindshorizontal<3>Frame Body Text (transblindshorizontal).
%\pause\transblindsvertical<4>Frame Body Text (transblindsvertical).
%\pause\transdissolve<5>Frame Body Text (transdissolve).
%\pause\transglitter<6>Frame Body Text (transglitter). etc.
%\end{frame}
%=======================================================================
%%%%%%%%%%%%%%%%%%%%%%%%%%%%%%%%%%%%%%%%%%%%%%%%%%%%%%%%%%%%%%%%%%%%%%%%%%%%%%%%%%

%%%%%%%%%%%%%%%%%%%%%%%%%%%%%%%%%%%%%%%%%%%%%%%%%%%%%%%%%%%%%%%%%%%%%%%%%%%%%%%%%%
\section{Gambar}
%=======================================================================
\begin{frame}[containsverbatim,allowframebreaks]{Gambar}

\begin{figure}
\begin{center}
\includegraphics[height=.65\textheight]{plotwithMath.pdf}
%\includegraphics<3>[height=.7\textheight]{whiteLandscape.jpg}
\end{center}
\caption{Gambar vektor (misal dalam format Postscript [PS atau EPS] atau PDF).}
\end{figure}

\begin{block}{}
\begin{verbatim}
\begin{figure}
\begin{center}
\includegraphics[height=.65\textheight]{plotwithMath.pdf}
\end{center}
\caption{Gambar vektor (misal dalam format Postscript [PS atau EPS]
         atau PDF).}
\end{figure}
\end{verbatim}
\end{block}

\begin{figure}
\begin{center}
\includegraphics[height=.65\textheight]{whiteLandscape.jpg}
\end{center}
\caption{Gambar \textit{raster} (misal dalam format JPG atau PNG).}
\end{figure}

\begin{block}{}
\begin{verbatim}
\begin{figure}
\begin{center}
\includegraphics[height=.65\textheight]{whiteLandscape.jpg}
\end{center}
\caption{Gambar \textit{raster} (misal dalam format JPG atau PNG).}
\end{figure}
\end{verbatim}
\end{block}

\end{frame}
%=======================================================================
%%%%%%%%%%%%%%%%%%%%%%%%%%%%%%%%%%%%%%%%%%%%%%%%%%%%%%%%%%%%%%%%%%%%%%%%%%%%%%%%%%

%%%%%%%%%%%%%%%%%%%%%%%%%%%%%%%%%%%%%%%%%%%%%%%%%%%%%%%%%%%%%%%%%%%%%%%%%%%%%%%%%%
%\section{Teorema}
%=======================================================================
%\begin{frame}[allowframebreaks]{Teorema}
%\label{theorem}
%\begin{theorem}
%In a right triangle, the square of hypotenuse equals
%the sum of squares of two other sides.
%\end{theorem}
%
%\begin{block}{Blok dengan judul manasuka}
%Test
%\end{block}
%
%\begin{algoblock}[Bisection]
%\begin{enumerate}
%\item Coba
%\item Cobi
%\end{enumerate}
%\end{algoblock}
%
%\begin{instance}
%Coba.
%\end{instance}
%
%\begin{instance}
%Lagi.
%\end{instance}
%
%\begin{exercise}
%Is $ 2 +2 = 4 $?
%\end{exercise}
%
%\begin{assignment}[Deadline: Today $ +\,2 $,  23:59 WIB]
%Prove that $ \sqrt{2} $ is an irrational number.
%\end{assignment}
%
%\begin{exercise}
%\lipsum[1]
%\theorembreak
%\lipsum[2]
%\end{exercise}
%
%\end{frame}
%=======================================================================
%%%%%%%%%%%%%%%%%%%%%%%%%%%%%%%%%%%%%%%%%%%%%%%%%%%%%%%%%%%%%%%%%%%%%%%%%%%%%%%%%%

%%%%%%%%%%%%%%%%%%%%%%%%%%%%%%%%%%%%%%%%%%%%%%%%%%%%%%%%%%%%%%%%%%%%%%%%%%%%%%%%%%
%\section{Salindia Bersambung-sambung}
%=======================================================================
%\begin{frame}[allowframebreaks]{Satu Pasal dalam Banyak Salindia}
%\lipsum
%\end{frame}
%=======================================================================
%%%%%%%%%%%%%%%%%%%%%%%%%%%%%%%%%%%%%%%%%%%%%%%%%%%%%%%%%%%%%%%%%%%%%%%%%%%%%%%%%%

%%%%%%%%%%%%%%%%%%%%%%%%%%%%%%%%%%%%%%%%%%%%%%%%%%%%%%%%%%%%%%%%%%%%%%%%%%%%%%%%%%
\section{Rujukan Internal}
%=======================================================================
\begin{frame}[containsverbatim,allowframebreaks]{Rujukan Internal}
\lipsum[3]

\bigskip

Lihat persamaan~(\ref{sqrtpi}). Untuk rincinya, sila lihat halaman~\ref{math}. Lihat pula Daftar Isi (halaman~\ref{contents}).

\begin{block}{}
\begin{verbatim}
Lihat persamaan~(\ref{sqrtpi}). Untuk rincinya, sila lihat
halaman~\ref{math}. Lihat pula Daftar Isi (halaman~\ref{contents}).
\end{verbatim}
\end{block}
\end{frame}
%=======================================================================
%%%%%%%%%%%%%%%%%%%%%%%%%%%%%%%%%%%%%%%%%%%%%%%%%%%%%%%%%%%%%%%%%%%%%%%%%%%%%%%%%%

%%%%%%%%%%%%%%%%%%%%%%%%%%%%%%%%%%%%%%%%%%%%%%%%%%%%%%%%%%%%%%%%%%%%%%%%%%%%%%%%%%
\section{Tabel}
%=======================================================================
\begin{frame}[allowframebreaks]{Tabel}

\begin{table}
\label{tabExample0}
\caption{Contoh tabel format baku: rapat kiri, tengah, \textit{merge columns}.}
        \begin{tabular}{|c|l|}
    \hline
    \textbf{No} & \multicolumn{1}{c|}{\bfseries Nama} \\\hline\hline
    1  & Aman \\\hline
    2  & Mana \\\hline
    3  & Anam \\\hline
    \multicolumn{2}{|c|}{Tak ada lagi kombinasi}\\\hline
\end{tabular}
\end{table}

\newpage

\begin{table}
\label{tabExample1}
\caption{Contoh tabel berwarna. Paket \textsf{colortbl} harus ada dan dipanggil sebelum/di luar \textit{document environment}.}
\begin{center}
\begin{tabular}{cll}
\rowcolor{Brown}\multicolumn{1}{c}{\textcolor{white}{No}} & \multicolumn{1}{c}{\textcolor{white}{Proxy function}} & \multicolumn{1}{c}{\textcolor{white}{Centroid}}\\
\rowcolor{Khaki}1 & Manhattan & median\\
\rowcolor{DarkKhaki}2 & Euclidean & mean
\end{tabular}
\end{center}
\end{table}
\end{frame}

\begin{frame}{Tabel dengan Lapisan Samar}

\begin{table}
\label{tabExample2}
\caption{Tabel dengan lapisan samar (hanya cocok untuk presentasi).}
\begin{center}
\begin{tabular}{l<{\onslide}!{\vrule}c<{\onslide<2->}c<{\onslide<3->}c<{\onslide<4->}c<{\onslide<5->}c}
\rowcolor{Brown}Class & A & B & C & D \\
\rowcolor{Khaki}X & 1 & 2 & 3 & 4 \\
\rowcolor{DarkKhaki}Y & 3 & 4 & 5 & 6 \\
\rowcolor{Khaki}Z & 5 & 6 & 7 & 8
\end{tabular}
\end{center}
\end{table}

\end{frame}
%=======================================================================
%%%%%%%%%%%%%%%%%%%%%%%%%%%%%%%%%%%%%%%%%%%%%%%%%%%%%%%%%%%%%%%%%%%%%%%%%%%%%%%%%%

%%%%%%%%%%%%%%%%%%%%%%%%%%%%%%%%%%%%%%%%%%%%%%%%%%%%%%%%%%%%%%%%%%%%%%%%%%%%%%%%%%
%\section{Gambar Berbayang Buram Menggunakan Tikz, bukan Fancybox}
%=======================================================================
%\begin{frame}[allowframebreaks]{Gambar Berbayang}
%\shadowimage[width=.61\linewidth]{whiteLandscape.jpg}{Winter. Picture put by \texttt{shadowimage} command (with shadow).}
%\end{frame}
%=======================================================================

%=======================================================================
\begin{frame}[plain]
\pause
\begin{figure}
\includegraphics[width=.8\linewidth]{whiteLandscape.jpg}
\pause
\caption{Winter. Picture put by \texttt{figure} environment (without shadow).}
\end{figure}
\end{frame}
%=======================================================================

%=======================================================================
\begin{frame}[plain]
\pause
\shadowimage[width=.7\linewidth]{plotwithMath.pdf}{Plot of $ x^3 + 3 $ with Mathematica, with shadow.}
\end{frame}
%=======================================================================

%=======================================================================
\begin{frame}[plain]
\incgraph[documentpaper][width=\paperwidth,height=1.05\paperheight]{whiteLandscape.jpg}
\end{frame}
%=======================================================================
%%%%%%%%%%%%%%%%%%%%%%%%%%%%%%%%%%%%%%%%%%%%%%%%%%%%%%%%%%%%%%%%%%%%%%%%%%%%%%%%%%

%%%%%%%%%%%%%%%%%%%%%%%%%%%%%%%%%%%%%%%%%%%%%%%%%%%%%%%%%%%%%%%%%%%%%%%%%%%%%%%%%%
%\section{Dua Kolom}
%=======================================================================
%\begin{frame}[allowframebreaks]{Dua Kolom dalam Satu Salindia}
%
%The line you are reading goes all the way across the slide.
%From the left margin to the right margin.  Now we are going
%to split the slide into two columns.
%
%\begin{columns}[T]
%\begin{column}{0.5\textwidth}
%Here is the first column.  We put an itemized list in it.
%\begin{itemize}
%\item This is an item
%\item This is another item
%\item Yet another item
%\end{itemize}
%\end{column}
%
%\begin{column}{0.45\textwidth}
%\begin{figure}
%\begin{center}
%\includegraphics[width=0.45\textwidth]{logoITB.png}
%\end{center}
%\caption{Here is the second column.  We will put a picture in it.}
%\end{figure}
%\end{column}
%\end{columns}
%
%The line you are reading goes all the way across the slide.
%From the left margin to the right margin.
%
%\begin{columns}[T]
%\begin{column}{0.5\textwidth}
%Here is the first column.  We put an itemized list in it.
%\begin{itemize}
%\item This is an item
%\item This is another item
%\item Yet another item
%\end{itemize}
%\end{column}
%
%\begin{column}{0.45\textwidth}
%\shadowimage[width=0.75\textwidth]{whiteLandscape.jpg}{Here is the second column.  We will put a picture in it.}
%\end{column}
%\end{columns}
%
%The line you are reading goes all the way across the slide.
%From the left margin to the right margin.
%
%\end{frame}
%=======================================================================
%%%%%%%%%%%%%%%%%%%%%%%%%%%%%%%%%%%%%%%%%%%%%%%%%%%%%%%%%%%%%%%%%%%%%%%%%%%%%%%%%%

%%%%%%%%%%%%%%%%%%%%%%%%%%%%%%%%%%%%%%%%%%%%%%%%%%%%%%%%%%%%%%%%%%%%%%%%%%%%%%%%%%
\section{Sitiran (Citation)}
%=======================================================================
\begin{frame}[containsverbatim,allowframebreaks]{Sitiran (Citation)}
Sitiran ke pustaka rujukan dapat dilakukan, misalnya \cite{siess2000} dan \cite{stepien2002}.

\begin{block}{}
\begin{verbatim}
Sitiran ke pustaka rujukan dapat dilakukan, misalnya \cite{siess2000}
dan \cite{stepien2002}.
\end{verbatim}
\end{block}

\newpage

\begin{block}{Isi file bibliografi (misal bernama `biblio.bib`)}
\begin{verbatim}
@article{siess2000,
    author = {L. Siess and E. Dufour and M. Forestini},
    title = {An Internet Server for Pre-Main Sequence Tracks of Low- and
        Intermediate-mass Stars},
    journal = {A \& A},
    year = {2000}, volume = {358}, pages = {593}}

@article{stepien2002,
    author = {K. St\c{e}pie\'{n}},
    title = {Spin-up of Be Stars in the Pre-Main Sequence Phase},
    journal = {A \& A},
    year = {2002}, volume = {383}, pages = {218}}
\end{verbatim}
\end{block}

\newpage

Dalam dokumen \LaTeX\ ditulis:

\begin{block}{}
\begin{verbatim}
\documentclass{article}
...
\usepackage[square]{natbib}             % bibliography style package
...
\begin{document}
...
\section{Pustaka}     % atau \chapter{Pustaka}

\bibliography{biblio} % karena file bibliografi bernama `biblio.bib`
\bibliographystyle{authordate1}
...
\end{document}
\end{verbatim}
\end{block}

\begin{block}{Tahap kompilasi menuju produksi PDF (yang dianjurkan)}
\begin{itemize}
\item \verb|pdflatex <nama-file-LaTeX.tex>|
\item \verb|bibtex <nama-file-LaTeX>| $\Longleftarrow$ tanpa ekstensi \verb|.tex|
\item \verb|pdflatex <nama-file-LaTeX.tex>|, lalu ulangi
\item \verb|pdflatex <nama-file-LaTeX.tex>|
\end{itemize}
\end{block}

\end{frame}
%=======================================================================
%%%%%%%%%%%%%%%%%%%%%%%%%%%%%%%%%%%%%%%%%%%%%%%%%%%%%%%%%%%%%%%%%%%%%%%%%%%%%%%%%%

%%%%%%%%%%%%%%%%%%%%%%%%%%%%%%%%%%%%%%%%%%%%%%%%%%%%%%%%%%%%%%%%%%%%%%%%%%%%%%%%%%
\section{Coding (verbatim atau listings)}
%=======================================================================
\begin{frame}[containsverbatim,allowframebreaks]{Coding (verbatim atau listings)}
\begin{verbatim}
#include <stdio.h>

/* using verbatim environment */
int main(){
    puts("Hello world!");
    return 0;
}
\end{verbatim}

\newpage

\textbf{\underline{Ditulis}:}

\bigskip

\verb|\begin{verbatim}|
\begin{verbatim}
#include <stdio.h>

/* using verbatim environment */
int main(){
    puts("Hello world!");
    return 0;
}
\end{verbatim}
\verb|\end{verbatim}|

\newpage

\begin{lstlisting}[style=CStyle]
#include <stdio.h>

/* using listings package */
int main(){
    puts("Hello world!");
    return 0;
}
\end{lstlisting}

\newpage

Paket \textsf{listings} harus ada dan dipanggil sebelum/di luar \textit{document environment}.

\textbf{\underline{Kode program ditulis}:}

\bigskip

\verb|\begin{lstlisting}[style=CStyle]|
\begin{verbatim}
#include <stdio.h>

/* using listings package */
int main(){
    puts("Hello world!");
    return 0;
}
\end{verbatim}
\verb|\end{lstlisting}|

\newpage

\textsf{PyStyle} \& \textsf{FortranStyle} untuk kode Python dan Fortran.

\begin{lstlisting}[style=PyStyle]
# This is python code
def main():
    print('Hello world!')
    return

if __name__ == '__main__':
    main()
\end{lstlisting}

\begin{lstlisting}[style=FortranStyle]
! This is Fortran code
program hello
print *, "Hello world!"
end program hello
\end{lstlisting}

\end{frame}
%=======================================================================
%%%%%%%%%%%%%%%%%%%%%%%%%%%%%%%%%%%%%%%%%%%%%%%%%%%%%%%%%%%%%%%%%%%%%%%%%%%%%%%%%%

%%%%%%%%%%%%%%%%%%%%%%%%%%%%%%%%%%%%%%%%%%%%%%%%%%%%%%%%%%%%%%%%%%%%%%%%%%%%%%%%%%
\section{Video}
%=======================================================================
\begin{frame}{Video (Klik Gambar atau Tulisan)}
\vfill

Tingkat kerumitan untuk multimedia (misalnya video) lebih tinggi, namun sudah biasa difasilitasi untuk presentasi (beamer) atau bahkan paper untuk jurnal.

Tautan di bawah ini hanya untuk menunjukkan saja, belum tentu bebas kesalahan.

\bigskip

\centering\movie[autostart,showcontrols,externalviewer]{\includegraphics[scale=0.2]{generalRelativity.png}}{visualizingGeneralRelativity.mp4}
\vfill
%options: autostart, loop, repeat, palindrome
%options: borderwidth, showcontrols, externalviewer
\end{frame}
%=======================================================================
%%%%%%%%%%%%%%%%%%%%%%%%%%%%%%%%%%%%%%%%%%%%%%%%%%%%%%%%%%%%%%%%%%%%%%%%%%%%%%%%%%

%%%%%%%%%%%%%%%%%%%%%%%%%%%%%%%%%%%%%%%%%%%%%%%%%%%%%%%%%%%%%%%%%%%%%%%%%%%%%%%%%%
\section{Audio}
%=======================================================================
\begin{frame}{Audio (Eksperimental)}
\vfill
\centering\sound[inlinesound,bitspersample=8,channels=2]{WAV Sample}{alarm.wav}
% File types depend on Acrobat Reader versions. Cannot be used with dvips/ps2pdf route
\vfill
%options: autostart, automute, loop, repeat
%options: inlinesound, channels (1), samplingrate (44100), bitspersample (16), encoding
\end{frame}
%=======================================================================
%%%%%%%%%%%%%%%%%%%%%%%%%%%%%%%%%%%%%%%%%%%%%%%%%%%%%%%%%%%%%%%%%%%%%%%%%%%%%%%%%%

%%%%%%%%%%%%%%%%%%%%%%%%%%%%%%%%%%%%%%%%%%%%%%%%%%%%%%%%%%%%%%%%%%%%%%%%%%%%%%%%%%
\section{\refname}
\pdfbookmark[1]{\refname}{references}
%=======================================================================
\begin{frame}{\refname, Untuk Menunjukkan Manfaat File Bibliografi}
\label{references}
\bibliography{biblio}
\bibliographystyle{authordate1}
\end{frame}
%=======================================================================
%%%%%%%%%%%%%%%%%%%%%%%%%%%%%%%%%%%%%%%%%%%%%%%%%%%%%%%%%%%%%%%%%%%%%%%%%%%%%%%%%%

%=======================================================================
%\begin{frame}[allowframebreaks]{Assignment} % or Homework. `allowframebreaks' and
                                             % `theorembreak' for long description
%\begin{frame}{Tugas}                    % or Homework
%\label{assignment}
%\pdfbookmark[1]{Assignment}{assignment} % or Homework
%\begin{assignment}[Deadline: Today $ +\,2 $,  23:59 WIB]
%Prove that $ \sqrt{2} $ is an irrational number.
%%\theorembreak
%%\lipsum[2]
%\end{assignment}
%\end{frame}
%=======================================================================

%=======================================================================
\begin{frame}[plain]
\vfill
\centering{\Huge\bfseries Terima kasih.}

\centering{\hyperlink{home}{\beamerreturnbutton{\bfseries Kembali ke Beranda}}}
\vfill
\end{frame}
%=======================================================================
%%%%%%%%%%%%%%%%%%%%%%%%%%%%%%%%%%%%%%%%%%%%%%%%%%%%%%%%%%%%%%%%%%%%%%%%%%%%%%%%%%

\end{document}
% END OF MAIN BODY %%%%%%%%%%%%%%%%%%%%%%%%%%%%%%%%%%%%%%%%%%%%%%%%%%%%%%%%%%%%%%%
