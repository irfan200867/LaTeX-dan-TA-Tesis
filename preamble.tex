% EDIT THIS FILE (MIHpreamble.tex)
\lstdefinestyle{CStyle}{
    backgroundcolor=\color{white},
    commentstyle=\itshape\color{DarkGreen},
    keywordstyle=\bfseries\color{red},
    numberstyle=\tiny\color{Gray},
    stringstyle=\color{DodgerBlue},
    basicstyle=\ttfamily,
    breakatwhitespace=false,
    breaklines=true,
    captionpos=b,
    keepspaces=true,
    numbers=left,
    numbersep=5pt,
    showspaces=false,
    showstringspaces=false,
    showtabs=false,
    tabsize=2,
    language=C
}
\lstdefinestyle{PyStyle}{
    backgroundcolor=\color{white},
    commentstyle=\itshape\color{DarkGreen},
    keywordstyle=\bfseries\color{red},
    numberstyle=\tiny\color{Gray},
    stringstyle=\color{DodgerBlue},
    basicstyle=\ttfamily,
    breakatwhitespace=false,
    breaklines=true,
    captionpos=b,
    keepspaces=true,
    numbers=left,
    numbersep=5pt,
    showspaces=false,
    showstringspaces=false,
    showtabs=false,
    tabsize=2,
    language=Python
}
\lstdefinestyle{FortranStyle}{
    backgroundcolor=\color{white},
    commentstyle=\itshape\color{DarkGreen},
    keywordstyle=\bfseries\color{red},
    numberstyle=\tiny\color{Gray},
    stringstyle=\color{DodgerBlue},
    basicstyle=\ttfamily,
    breakatwhitespace=false,
    breaklines=true,
    captionpos=b,
    keepspaces=true,
    numbers=left,
    numbersep=5pt,
    showspaces=false,
    showstringspaces=false,
    showtabs=false,
    tabsize=2,
    language=Fortran
}
\usetikzlibrary{shadows,calc}
\bibpunct{(}{)}{;}{a}{}{,}              % bibliography style format:
%         author [year]
% If you want to use Bahasa Indonesia
\usepackage[bahasa]{babel}
\uselanguage{Bahasa}
\languagepath{Bahasa}
\deftranslation[to=Bahasa]{Theorem}{Teorema}
\deftranslation[to=Bahasa]{Corollary}{Akibat}
\deftranslation[to=Bahasa]{Definition}{Definisi}
\deftranslation[to=Bahasa]{Example}{Contoh}
\deftranslation[to=Bahasa]{Proof}{Bukti}

% ITB COLORS:
\definecolor{ITB}{RGB}{40, 56,144} % note the uppercase RGB
\definecolor{G10}{RGB}{16,116,188} % note the uppercase RGB

\usetheme{Warsaw}
\useoutertheme{infolines,shadow}
\useinnertheme{rounded}
\usefonttheme{professionalfonts,structurebold}
\usecolortheme[named=ITB]{structure}
\setbeamertemplate{items}[ball]                  %default,ball,circle,rectangle
\setbeamertemplate{blocks}[rounded][shadow=true]
\setbeamertemplate{navigation symbols}{}
\setbeamertemplate{theorems}[numbered]
\setbeamertemplate{caption}[numbered]
%\setbeamertemplate{frametitle}[default][left]   %unfortunately, shadow disappear
\setbeamercolor{normal text}{fg=black}
%\setbeamertemplate{background canvas}{\includegraphics
%    [width=\paperwidth,height=\paperheight]{whiteLandscape.jpg}}
\setbeamercovered{transparent}
\usebackgroundtemplate{%
    \begin{tikzpicture}[remember picture,overlay]\node[opacity=0.01,inner sep=0] at (current page.center) {\includegraphics[scale=0.15]{logoITB}};\end{tikzpicture} % best opacity = 0.025
}
\hypersetup{colorlinks=true,allcolors=FireBrick,bookmarksopenlevel=1}
% \newtheorem{teorema}{Teorema} % theorem environment using Bahasa
% SOME NEW THEOREM/BLOCK ENVIRONMENTS %%%%%%%%%%%%%%%%%%%%%%%%%%%%%%%%%%%%%%%%%%%%
\newtheorem{instance}{Example}
\newcommand*{\theorembreak}{\usebeamertemplate{theorem end}\framebreak\usebeamertemplate{theorem begin}}
% Example or instance %%%%%%%%%%%%%%%%%%%%%%%%%%%%%%%%%%%%%%%%%%%%%%%%%%%%%%%%%%%%
\BeforeBeginEnvironment{instance}{
    \setbeamercolor{block title}{use=example text,fg=white,bg=Brown!50!black}
    \setbeamercolor{block body}{parent=normal text,use=block title example,bg=block title example.bg!10!bg}
}
\AfterEndEnvironment{instance}{
    \setbeamercolor{block title}{use=structure,fg=white,bg=structure.fg!75!black}
    \setbeamercolor{block body}{parent=normal text,use=block title,bg=block title.bg!10!bg}
}
\newtheorem{exercise}{Exercise}
% Exercise %%%%%%%%%%%%%%%%%%%%%%%%%%%%%%%%%%%%%%%%%%%%%%%%%%%%%%%%%%%%%%%%%%%%%%%
\BeforeBeginEnvironment{exercise}{
    \setbeamercolor{block title}{use=example text,fg=white,bg=DarkKhaki!50!black}
    \setbeamercolor{block body}{parent=normal text,use=block title example,bg=block title example.bg!10!bg}
}
\AfterEndEnvironment{exercise}{
    \setbeamercolor{block title}{use=structure,fg=white,bg=structure.fg!75!black}
    \setbeamercolor{block body}{parent=normal text,use=block title,bg=block title.bg!10!bg}
}
\newtheorem{assignment}{Assignment/Homework}
% Exercise %%%%%%%%%%%%%%%%%%%%%%%%%%%%%%%%%%%%%%%%%%%%%%%%%%%%%%%%%%%%%%%%%%%%%%%
\BeforeBeginEnvironment{assignment}{
    \setbeamercolor{block title}{use=example text,fg=white,bg=black}
    \setbeamercolor{block body}{parent=normal text,use=block title example,bg=block title example.bg!10!bg}
}
\AfterEndEnvironment{assignment}{
    \setbeamercolor{block title}{use=structure,fg=white,bg=structure.fg!75!black}
    \setbeamercolor{block body}{parent=normal text,use=block title,bg=block title.bg!10!bg}
}
\newtheorem{algoblock}{Algorithm}
% Example or instance %%%%%%%%%%%%%%%%%%%%%%%%%%%%%%%%%%%%%%%%%%%%%%%%%%%%%%%%%%%%
\BeforeBeginEnvironment{algoblock}{
    \setbeamercolor{block title}{use=example text,fg=white,bg=G10}
    \setbeamercolor{block body}{parent=normal text,use=block title example,bg=block title example.bg!10!bg}
}
\AfterEndEnvironment{algoblock}{
    \setbeamercolor{block title}{use=structure,fg=white,bg=structure.fg!75!black}
    \setbeamercolor{block body}{parent=normal text,use=block title,bg=block title.bg!10!bg}
}
%%%%%%%%%%%%%%%%%%%%%%%%%%%%%%%%%%%%%%%%%%%%%%%%%%%%%%%%%%%%%%%%%%%%%%%%%%%%%%%%%%

% Drop image shadow with gaussian blur (using Tikz, not fancybox) %%%%%%%%%%%%%%%%
% some parameters for customization
\def\shadowshift{3pt,-3pt}
\def\shadowradius{6pt}
\colorlet{innercolor}{black!60}
\colorlet{outercolor}{gray!05}
% this draws a shadow under a rectangle node
\newcommand\drawshadow[1]{
    \begin{pgfonlayer}{shadow}
        \shade[outercolor,inner color=innercolor,outer color=outercolor] ($(#1.south west)+(\shadowshift)+(\shadowradius/2,\shadowradius/2)$) circle (\shadowradius);
        \shade[outercolor,inner color=innercolor,outer color=outercolor] ($(#1.north west)+(\shadowshift)+(\shadowradius/2,-\shadowradius/2)$) circle (\shadowradius);
        \shade[outercolor,inner color=innercolor,outer color=outercolor] ($(#1.south east)+(\shadowshift)+(-\shadowradius/2,\shadowradius/2)$) circle (\shadowradius);
        \shade[outercolor,inner color=innercolor,outer color=outercolor] ($(#1.north east)+(\shadowshift)+(-\shadowradius/2,-\shadowradius/2)$) circle (\shadowradius);
        \shade[top color=innercolor,bottom color=outercolor] ($(#1.south west)+(\shadowshift)+(\shadowradius/2,-\shadowradius/2)$) rectangle ($(#1.south east)+(\shadowshift)+(-\shadowradius/2,\shadowradius/2)$);
        \shade[left color=innercolor,right color=outercolor] ($(#1.south east)+(\shadowshift)+(-\shadowradius/2,\shadowradius/2)$) rectangle ($(#1.north east)+(\shadowshift)+(\shadowradius/2,-\shadowradius/2)$);
        \shade[bottom color=innercolor,top color=outercolor] ($(#1.north west)+(\shadowshift)+(\shadowradius/2,-\shadowradius/2)$) rectangle ($(#1.north east)+(\shadowshift)+(-\shadowradius/2,\shadowradius/2)$);
        \shade[outercolor,right color=innercolor,left color=outercolor] ($(#1.south west)+(\shadowshift)+(-\shadowradius/2,\shadowradius/2)$) rectangle ($(#1.north west)+(\shadowshift)+(\shadowradius/2,-\shadowradius/2)$);
        \filldraw ($(#1.south west)+(\shadowshift)+(\shadowradius/2,\shadowradius/2)$) rectangle ($(#1.north east)+(\shadowshift)-(\shadowradius/2,\shadowradius/2)$);
    \end{pgfonlayer}
}
% create a shadow layer, so that we don't need to worry about overdrawing other things
\pgfdeclarelayer{shadow}
\pgfsetlayers{shadow,main}
\newsavebox\mybox
\newlength\mylen
\newcommand\shadowimage[3][]{%
    \setbox0=\hbox{\includegraphics[#1]{#2}}
    \setlength\mylen{\wd0}
    \ifnum\mylen<\ht0
    \setlength\mylen{\ht0}
    \fi
    \divide \mylen by 120
    \def\shadowshift{\mylen,-\mylen}
    \def\shadowradius{\the\dimexpr\mylen+\mylen+\mylen\relax}
    \begin{figure}
        \begin{tikzpicture}
            \node[anchor=south west,inner sep=0] (image) at (0,0) {\includegraphics[#1]{#2}};
            \drawshadow{image}
        \end{tikzpicture}
        \caption{#3}
\end{figure}}
%%%%%%%%%%%%%%%%%%%%%%%%%%%%%%%%%%%%%%%%%%%%%%%%%%%%%%%%%%%%%%%%%%%%%%%%%%%%%%%%%%